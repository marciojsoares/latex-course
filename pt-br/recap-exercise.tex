\documentclass[12pt]{article}
\usepackage[brazil]{babel}
 \usepackage[T1]{fontenc} % Adicionado para poder usar os Caracteres Portugueses
\usepackage[utf8]{inputenc} % Adicionado para poder usar os Caracteres Portugueses

\usepackage{url}

\begin{document}
Dez Segredos Para uma Boa Apresentação

Autor: Você

Introdução

O Texto constante neste exercício é uma versão traduzida e resumida do excelente artigo da autoria de Mark Schoeberl e Brian Toon:
\url{http://www.cgd.ucar.edu/cms/agu/scientific_talk.html}

Os Segredos

Compilei esta lista pessoal de "Segredos" depois de ouvir apresentações eficazes e ineficazes de diversos oradores. Esta lista não é compreensiva - De certeza que coisas ficaram de fora. Mas, esta lista contêm 90% do que tu deves saber e deves fazer.

1) Prepara o teu material cuidadosamente e logicamente. Conta uma história.

2) Pratica o teu discurso. Falta de preparação não é uma desculpa.

3) Não tenhas demasiado material. Bons Oradores apresentam um ou dois pontos principais e focam-se neles.

4) Evita equações. diz-se que por cada equação na tua apresentação, o número de pessoas que te compreendem passa para metade. Isto é, sendo q o número de equações na apresentação, o número de pessoas que percebem a apresentação é dado por:

n = gamma (1/2) elevado a q

onde gama é a constante de proporcionalidade.

5) A conclusão deve conter apenas os pontos chave. As pessoas não se lembram de mais que 2 ou 3 coisas de uma apresentação, principalmente se tiverem ouvido várias apresentações em grandes conferências.

6) Fala para audiência e não para o ecrã. Um dos erros mais comuns é o orador falar virado de costas para audiência. 

7) Evita fazer sons que distraiam audiência. Evita os "Ummm" ou "Ahhh" entre frases.

8) Melhora os teus gráficos. Uma pequena lista de dicas para melhorar gráficos numa apresentação:

* Usa uma fonte grande.

* Mantém os gráficos simples. Não mostres gráficos que não precisas.

* Usa cores.

9) Sê pessoal a responder a questões.

10) Usa humor sempre que possível. É fascinante como uma piada seca consegue fazer as pessoas rir numa conferência cientifica.



Referências 


[1] W. Boyce e R. C. DiPrima. Equações Diferenciais elementares e Problemas de Valor de Contorno. 10a ed. Rio de Janeiro: LTC, 2010, p. 624.

[2] Gilbert Strang. Introdução à Álgebra Linear. 4a ed. Rio de Janeiro: LTC, 2013, pp. 166-167.

[3] Wagner Rodrigues Valente. Há 150 Anos Uma Querela sobre a Geometria Elementar no Brasil: algumas Cenas dos Bastidores da Produção do Saber Escolar. Em: BOLEMA 12 (13 1999), pp. 44–61.


\end{document}